\chapter{序論}
論文は序論のようなもので始める。タイトルは序論でも序言でもはじめにでもいいけど、『序論』で始めたら『結論』で終わり、『序言』で始めたら『結言』で終わるようにする。『はじめに』なら『おわりに』で終わる。『序論』で始まって『おわりに』でおわるとか、そういうちぐはぐなのはだめ。
ここでは序論として書く。序論では、研究の背景やら目的やらを書くのが普通。今はテンプレートの説明なので、大して書くことは無い。

\section{背景}
テンプレートの見た目がどうなるかを見せなければならないので、ちょっと長めの文を書く。
ぼくは別に\LaTeX に明るいわけではなくて、この研究室に所属してから初めて触った程度。四年生になってぼく自身が卒業論文を書くことになって、先生は\LaTeX を推奨していたんだけど、テンプレートありますかって聞いたら特にないから作ってほしいとのことだったので、じゃあ作りますよ、という流れ。ぼく自身が使いやすいように、自分が使いながらいろいろ改良をして、こうして公開している。
作成にあたっては、先輩方の卒業論文や主にぐーぐる先生を活用したインターネット上の情報を参考にした。
ただ、卒業論文の体裁は、それぞれの研究室の文化や、担当の指導教員のこだわりも強く影響することも事実。このテンプレートは、『ぼくが所属していた研究室』という、ごくごく限定的でローカルな仕様に沿ったフォーマット——より正確に言ば『ぼくが所属していた研究室ではNGではなかった』フォーマット——というだけのもの。そのあたり、承知の上で使ってほしい。
他の研究室で使う場合は、指導教員の許可を仰ぐほうが確実。

\subsection{ほげ}
書きましょう
