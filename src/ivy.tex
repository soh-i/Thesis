\chapter{検出手法の開発と実装}

\section{研究概要}
RNA editingとは、転写物へ位置特異的に一塩基置換を引き起こす転写後修飾の一種として知られ、アデニン (A)からイノシン (I)へのA-to-I editingがヒトやマウス、ショウジョウバエから多数報告されている。このA-to-I editingはADARと呼ばれる二本鎖RNA結合タンパク質によって触媒されることが知られており、翻訳の段階で置換されたイノシンはグアノシンとして認識されるため、editingを受けた転写物は翻訳の過程において、非同義置換によるスプライシングサイトの変化やタンパク質の高次構造の変化、microRNAへのeditingを介した遺伝子発現の抑制など転写調節に幅広く関与していることが報告されている。近年、RNA-seqデータを用いたゲノムワイドなeditingサイトの同定が多数の組織およびセルラインを用いて行われ、ヒトでは数万箇所のeditingサイトが報告されている。RNA editingサイトはゲノムと転写物の一塩基のミスマッチとして検出可能だが、シーケンシングやマッピングに起因した擬陽性を多く含むため、真のeditingサイトと擬陽性を高精度に分離する検出手法がこれまで多く提案されている。しかしながら、解析に使用された手法の多くはソフトウェアとして公開されておらず、RNA-seqデータを対象としたeditingサイトの検出ソフトウェアは現時点で一つ存在するのみである。そこで本研究では、既存のソフトウェアよりも高速かつ低メモリで動作し、アラインメントデータへの統計的なフィルタリング手法、実験デザインを考慮した解析を可能にするRNA editingサイトの検出パッケージの開発を行った。本パッケージは、既存のソフトウェアと比較して高速か低メモリで動作し、付属するベンチマーキングツールによって、検出したeditingサイトの検出精度を定量的に評価することを可能にした。尚、本パッケージは、GPLの元、オープンソースのフリーウェアとして公開することを予定している。

\section{開発の背景}

\section{要求分析}

\section{検出手法の特徴}

\section{実装}
\subsection{設計したシステム}
\subsection{データの入出力}
\subsection{並列化}
\subsection{使用例}

\section{ベンチマーク}
\subsection{使用したデータ}

\section{結果}
\subsection{ベンチマーク結果}
\subsection{解析結果}

\section{議論}
\subsection{本手法により同定されたeditingサイト}
\subsection{本手法の有用性}


