\chapter*{謝辞}
\addcontentsline{toc}{chapter}{謝辞}
冨田研究室に配属されてからのおよそ3年半、研究室での会話や進捗のミーティング、学会発表など様々な場面で、厳しく有用な指摘があり、多くの勇気づけられる助言があった。本研究を行うにあたり、関わって頂いた多くの人に心からお礼を申し上げたい。
\par
慶應義塾大学 政策・メディア研究科 荒川和晴講師には、学部1年秋学期より一貫して研究に関する指導をして頂きました。基本的に楽をして解析を済ませようとする怠惰な僕に対し、真っ当な解析方法を提示される荒川さんの指摘は、研究の質を少しずつ上げてくれるものでした。加えて、毎日の研究をいかに進めていくかというマネージメントなど、研究の方法論についても多くの気付きをもたらしたかけがえの無い経験でした。またスランプに陥った時も、なんとか次の研究につながるアドバイスを頂きました。深く感謝致します。
\par
高校二年次に、SBP (Super bioscience program)に参加したきっかけが冨田勝教授との出会いでした。冨田さんとの出会いがなければ、SFCへ進学していなかったと思います。間違いなく人生における重要な出会いでした。所属した学部2年生までは悲しい程にプログラムが書けなくてこの先研究ができるのか心配なくらいでしたが、続けていくうち次第に自分の得意と言えるスキルへ変化していった感触があります。当時、卒業論文でバイオインフォマティクスにおけるソフトウェア開発を行うことは全く想像できませんした。こうった研究領域への関心が変化していくことに寛容な雰囲気がある冨田研で研究できたことは幸運でした。非常に恵まれた研究の機会と環境を与えて頂いた冨田さんにお礼申し上げます。同時に、秘書の見上さん、水上さん、平本さんには大学院の出願や学会発表、鶴岡での春・夏プロジェクトにおいて、大変にお世話となりました。感謝致します。
\par
基本的に研究は三歩コマが進んだら二歩、悪いと四歩くらい後退するものでありましたが、それでもなんとかやってこられたのは同じような境遇の同期が周りにいたからだと思います。研究会の同期には公私(多くは不可分だが)問わず助けられました。土岐珠未氏や真流玄武氏とは正月を返上して昼夜問わず研究したし、今井淳之介氏にも世話になった。鶴岡に行ってしまった梅田栄美氏や臼井優希氏にも大きな感謝を表したいです。早期卒業した川崎翠氏にも遥かロンドンに向かって大きな感謝を送りたい。こういった多彩な顔ぶれの同期がいなかったら、研究生活はもっとつまらないものだったと想像します。本当にありがとう。
\par
所属するG-languageグループの皆様には本当に感謝しています。吉田勇太氏、そして特にグループの5人の後輩たちは、いつも後輩らしくないでかい態度で接してくれましたから、僕にとっても同期が増えたようで、なかなかよい関係を築くことができました。後輩たちとの分け隔てない関係は、楽しく気の休まるものでした。大下和希氏には、卒業後も付き合いがあるほど本当によくしていただきました。実装や解析に詰まった数えきれない回数、多くの時間を割いて一緒に解決して頂き、本当に感謝しています。
\par
野崎慎氏には、研究に関する様々なアドバイスを頂きました。RNA編集の研究は泥沼やでと最初に脅されましたが(実際そうでしたが)、解析に関するアドバイスや議論は常に本質的で有益でした。また、香川県に行くにあたり、はしごすべき讃岐うどんの名店を丁寧に教えて頂きました。どの店も最高の讃岐うどんでした。新土優樹氏からは研究に対する姿勢や研究の進め方などを影ながら多くを参考にさせて頂きました。玉木聡志氏とは技術的な話をするのが僕はすごく楽しかったです。いつでも楽観的のようにみえる玉木さんの生き方には何度も救われるようでした。松井求氏には、鶴岡に行った時などにRNA編集に関する議論などができ、解析に関する幾つかの重要な指摘をして頂いたほか、根気強く研究に打ち込む姿には強い感銘を受けました。
\par
本卒業研究は、SFCの卒業プロジェクトの履修者を対象とした株式会社GREEの副社長 山岸広太郎氏の寄付による第一回山岸学生支援プロジェクトに採択して頂き、研究開発および学会発表が可能となりました。この場をお借りして感謝を申し上げます。山岸学生支援プロジェクトでは、専門が全く異なる方へ自身の研究と意義を伝えることの困難さに直面することができた他に、採択された他の学生とその指導教官の先生と交流する機会にも恵まれた大変に貴重な機会でした。
\par
2013年夏に開催されたE-Cell sprint 2013では、理化学研究所 生命システム研究センターの高橋恒一博士、海津一成博士を始めとした細胞シミュレーションを専門とした方たちと夜通し議論し、コードを書くことができた経験は、僕にとって象徴的な出来事となりました。これまで塩基配列など実データありきの研究以外を経験してこなかった僕にとって、生命現象すらも高度に抽象化されたプログラミングによって表現できるのだという驚きは今も鮮明に記憶しています。shafiさんには個人的にグルメ情報や進路について多くの助言を頂いた他、一杯のコーヒーのために丹沢まで2時間かけて名水を汲みに行ったりもしました。E-Cell sprintには学外からの学生も参加し、その後も仲良くできているのは本当に貴重です。このような機会を提供してくださった内藤泰宏准教授にも感謝致します。内藤さんには『細胞の物理生物学』の輪読会でお世話になった他、何気ない会話の中に大きな知の体系が見え隠れする瞬間があり、その度に強い感銘を受けました。
\par
幸いにも数回の学会発表の機会に恵まれ、その度に学外の先生や学生の方とのよき出会いがありました。二度にわたって参加させて頂いたNGS現場の会や分子生物学学会では、ポスター発表や懇親会において学外の多くの学生の方と知り合うことができました。彼らからは、同年代として多くの刺激や進学に関するアドバイスをもらい、研究の励みとなりました。また、山口大学 鈴木治夫博士には、学会などでお会いする度に優しい助言を頂きました。深謝致します。
\par
三木研研究室 修士課程の小澤みゆき氏および武藤研究室 博士課程の中島博敬氏の両氏には、研究分野が離れているにも関わらず、非常に仲良くしていただき、研究を遂行するにあたって励みとなっていたことをここで表明させて頂きます。
\par
最後になりましたが、大学生活を通して一貫して好き勝手にやらせてくれた両親と家族へ深く感謝し、卒業論文の締めとさせて頂きます。
