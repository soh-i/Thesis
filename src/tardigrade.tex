\chapter{クマムシにおけるRNA editingサイトの解析}

\section{研究概要}
一時的に代謝の停止した乾眠状態となり、乾眠状態からは吸水によって生命活動を再開する。このような可逆性を持つ乾眠状態は同時に、タンパク質の変性と凝集、核酸の損傷などを引き起こすため、多くの乾眠動物では遺伝子発現の様式を大規模に変化させ、保護物質としてのトレハロースやLEAタンパクを蓄積させている。ところが、{\it R. varieornatus}は急速に乾眠状態へと移行するため、活動状態と乾眠状態では、遺伝子発現変動が僅かである特徴が明らかになってきた。このような急速に移行する {\it R. varieornatus}の乾眠は、定常的に発現している遺伝子群に加え、RNA editingを始めとする転写産物への修飾や、タンパク質のリン酸化修飾などによる制御が示唆されている。そこで本研究では、RNA-seqデータを用いて{\it R. varieornatus}におけるRNA editingサイトのゲノムワイドな検出を行った。RNA editingサイトはRNA-seqリードをゲノムにマッピングした際の変異箇所として検出できるが、その多くは様々なバイアスに起因した擬陽性を示す。これらを複数の統計手法を組み合わせることで取り除いた結果、{\it R. varieornatus}における全ての状態のトランスクリプトームから、他の真核生物と同様にA-to-G editingが優勢であることが示された。A-to-G editingサイトは、熱ショックタンパク質 (DnaJ)など分子シャペロンとして機能するタンパクから同定されており、{\it R. varieornatus}におけるRNA editingによる乾眠の制御機構が示唆された。

\section{背景}
\section{解析データと手法}
\subsection{解析データ}
\subsection{リードのマッピング}
\subsection{統計的フィルタリングを用いたRNA editingサイトの検出}
\subsection{プロテオームデータの解析}
\subsection{ADARホモログの予測}
\subsection{検出手法の精度検証}
\section{結果}
\subsection{検出されたサイト}
\subsection{Editingサイトの特徴解析}
\subsection{同定手法の精度}
\subsection{同定されたADARホモログ}
\subsection{ペプチド断片から見出されたアミノ酸変異}
\section{議論}