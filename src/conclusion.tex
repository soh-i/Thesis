\chapter{結論}
本論文は、超並列シーケンサーと呼ばれる高出力の塩基配列決定技術によって得られた大規模な塩基配列データから、RNA編集サイトの検出ソフトウェアの開発ならびに、情報学的な解析に関する内容を扱った。各章での議論を簡単にまとめた後、超並列シーケンサーデータを用いることによって解くべきRNA編集に関する問題を議論する。
\par
二章では、既存のRNA編集サイトの検出手法を統一的な指標を導入することにより、異なる検出手法の比較を可能にした。異なる検出手法の比較から、検出されたサイト周辺のリアラインメントや反復配列の除外など情報学的なフィルタリング手法に加えて、Strand specific RNA-seqや\textit{Adar}のノックダウン株のシーケンシグなど実験デザインの重要性が示唆された。この成果は、より精度の高い編集サイトの検出に貢献することが期待された。
\par
三章では、多くの研究では提案手法の実装がない状況において、超並列シーケンスデータを用いたRNA編集サイトの検出を目的としたソフトウェア・パッケージIvyの設計と実装を行い、オープンソースのフリーウェアとして公開した。Ivyは、RNA編集サイトの検出ツールおよび検出精度を評価することのできるベンチマークツールが同梱されたソフトウェア・パッケージである。公共データベースにおいて公開されているRNA-seqデータを用いて他のソフトウェアとの性能比較を行ったところ、Ivyは既存のRNA編集サイトの検出ソフトウェアと比較して、同等のメモリ効率ながら2倍程度高速に動作することが示された他、高い再現率を示すことが明らかとなった。
\par
四章では、ヨコヅナクマムシと呼ばれる乾眠動物のRNA-seqデータを用いてRNA編集サイトの検出を行った。非モデル生物であるため、完全にアセンブルされたゲノムがないといった解析の難しさがあったが、ADARを発現量の定量および機能ドメインレベルで機能推定を行った結果、ヨコヅナクマムシにおいてもA-to-I編集は起こりうる可能性を示すことができた。検出されたRNA編集サイトの中には、熱ショックからのタンパクの保護に関わる因子への編集が観察されたことから、今後は実験的な検証と共に、RNA編集と乾眠の関係性についての解析が進められることが期待された。\\
\par
今日、超並列シーケンサーは年々高性能化しており、一度のランで得られる総リード数は、ムーアの法則を凌駕している。今後、RNA-seqデータのエラー率は減少すると同時により高いカバレッジのデータを取得可能となることは容易に想像できる。このことは、RNA編集サイトにおける編集の頻度を定量的に議論することを可能にすると考えられる。
\par
RNA-seqデータから十分な精度で編集率を推定することが可能になると、\textit{Adar}
の発現量と編集率の関係性が明らかとなるだろう。一部の遺伝子に関しては\textit{Adar}の発現量と編集率の間には一貫した相関性が見られないことが報告されており \citep{Dominissini:2011aa}、単純に細胞内におけるADARの発現量と二本鎖RNAとの結合確率の上昇が編集率の増加へ寄与しないことは興味深く、ADARへの他の制御機構が機能している可能性がある。他にも、AluなどレトロトランスポゾンにおけるRNA編集の位置と頻度がどの程度制御されているのかは、内在性siRNAの産生にも関与することから興味深い。レトロトランスポゾン領域におけるランダムに起きているように見えるA-to-I編集サイトに関しても、ADARの発現量(≒細胞内濃度)と編集率の関係を明らかにすることで、背後に何らかの制御を見出すことが可能だろう。
\par
超並列シーケンサーから得られるRNA編集サイトに関する定量的なデータは、ADARの新たな機能と制御機序に関する問題を解く手がかりとなるだろう。本論文において議論してきた高精度かつ高速なRNA編集サイトの検出ソフトウェアの開発が、RNA編集の生物学的意味とADARを通した転写の調節機序の更なる解明へ貢献できることを期待したい。
