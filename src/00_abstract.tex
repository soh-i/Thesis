\begin{jabstract}
RNA編集とは、DNAがRNAへ転写される段階で起こる転写後修飾の一種である。修飾を受けたRNAはゲノムと異なる遺伝情報を持ち、タンパク質機能の多様化や、他の非翻訳RNAとの相互作用を介した遺伝子の発現制御へも関与する。ここ数年、超並列シーケンサーと呼ばれる高出力な塩基配列決定技術が普及し、ヒトやマウスなど高等真核生物に発現するRNAは、高頻度で編集を受けていることが明らかとなってきた。しかしながら、超並列シーケンサーデータを用いたRNA編集サイトの検出手法は、ソフトウェアとしての実装が現在では一つしかなく、その機能は解析に十分であるとは言い難い。そこで本研究は、再現率と適合率と呼ばれる指標を導入することにより、既存のRNA編集サイトの検出手法の比較を可能にし、高精度な検出手法と擬陽性を減少させるフィルタリング手法および実験デザインに関する議論を得た。その知見をもとに高精度かつ高速なRNA編集サイトの検出ソフトウェアIvyの開発を行った。開発したRNA編集サイトの検出ソフトウェアをグリア芽細胞腫由来のRNA-seqデータへ適用したところ、既存のソフトウェアと比較して同等のメモリ効率で2倍程度高速に動作することが確かめられたほか、全ての染色体において高い再現率を示す手法であることが示された。本研究は、超並列シーケンサーデータを用いたRNA編集サイトの高精度かつ高速な検出手法の開発に貢献することが期待される。
\end{jabstract}

\begin{flushright}
	\par
	2014年1月20日
\end{flushright}