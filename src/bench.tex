\chapter{検出手法の定量的評価}

\section{研究概要}
RNA編集は、転写物の塩基配列を変化させる転写後修飾であり、高等真核生物ではADARによるアデニンからイノシンへのA-to-I編集が最もよく知られる。近年、ゲノムワイドなA-to-I編集サイトの研究から、ヒトでは数千の編集サイトが同定されたと報告されている。RNA編集サイトはゲノムと転写物の一塩基ミスマッチとして検出されるが、シーケンシングやマッピングに起因した擬陽性を多く含む。そのため、真の編集サイトとエラーに起因した擬陽性を高精度に分離させる検出手法がこれまで多く開発されてきたが、開発された手法における検出精度の定量的なベンチマークは行われていない。そこで本研究は、ヒト・マウス・ショウジョウバエのRNA-seqデータを用いた既存の検出手法のベンチマークを行った。ベンチマーキングには、再現率/適合率といった指標を導入することにより、各手法の検出能の定量的な比較を行った。その結果、シーケンシング手法など実験デザイン及び検出手法ごとの特徴を明らかにした。この結果を交えながら、どのような検出パラメータがRNA編集サイトの高精度な検出に寄与するのかについて議論したい。

\section{背景}

\section{手法と対象}
\subsection{定量的な指標の導入}
\subsection{対象とした検出手法}
\subsection{正解セットの構築}

\section{ベンチマーク結果}

\section{議論}
\subsection{導入した指標の妥当性}
\subsection{高精度な検出に寄与するパラメータの探索}

