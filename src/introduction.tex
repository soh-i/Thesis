\chapter{序論}

\section{はじめに}
真核生物における遺伝子発現とその背後に潜む制御機構に関する研究は、ENCODEやFANTOMといった国際プロジェクトによって巨大なオミックスデータが測定され、転写の全体像が徐々に明らかになりつつある。またここ数年の間に、一細胞でのトランスクリプトーム解析が技術的に可能となり、細胞の分化や発生段階における遺伝子発現ゆらぎ(heterogenity)とった、

\section{RNA editing}
\subsection{真核生物における修飾の重要性}
転写はゲノム上にコードされた遺伝情報を正確にコピーする機構である。転写は核内で起こる生化学反応であり、RNAポリメラーゼよって触媒される。転写された一本のpre-mRNAはスプライシングを受けイントロン領域が切りだされ、ポリアデニル化と5'キャップ付加を受けて成熟した後に、核膜孔を通して細胞質へ輸送される、というのが転写機構の大まかな素描である。ヒトゲノムの解読前、遺伝子数はおよそ10万個だと見積もられていたが、解読の完了によって2.5万個程度に大きく下方修正され、直感的な生命の複雑性とゲノムサイズや遺伝子数には相関関係が見られないことは今日に繰り返すまでもない。今日までの研究によって、真核生物の見せる多様性で複雑なシステムの多くは、転写後および翻訳後の転写物やタンパク質への修飾によって大部分が担保されていることが明らかとなってきた。細胞の内外の情報伝達の多くは、キナーゼによるタンパク質のカスケード的なリン酸化が引き金となり、転写因子が特定の遺伝子発現を制御する。このように、真核生物においては、前述した転写の素描に加え、有限個の遺伝子にその多様性を規定されながらも、転写後修飾および翻訳後修飾によって、転写物レベルあるいはタンパク質レベルでの多様性や複雑性を保証する戦略を進化させてきたと言えるだろう。
\par
本研究は転写後修飾のとして最もよく知られるRNA editingという修飾現象に注目をおいている。
タンパク質に修飾を加えることによって、有限個の遺伝子から転写物、タンパク質レベルでの多様性を生み出す戦略を進化させてきた。
\par
RNA editingは転写物への一塩基修飾を指し、から初めて報告された。真核生物では、アデニン(A)からイノシン(I)へ修飾されるA-to-I editingの他に、シトシン(C)からチミン(T)への修飾がこれまで報告されている。植物においてはT-to-C editing、ヒトやマウス、ショウジョウバエなど高等真核生物においてはA-to-I editingが優勢を占めることが多くの研究から明らかになっている。

\subsection{ADARのドメイン構造と生化学的機序}
ADAR (adenosine deaminase acting on RNA)は、二本鎖RNA結合タンパクの一種として知られ、相補鎖を形成した二本鎖RNAと選択的に結合し、イノシンへの置換を触媒する。

以下にADARによるRNA editingの概略を示す。

\subsection{細胞内局在と遺伝子発現}

\section{真核生物におけるRNA editing}
\subsection{コード領域におけるeditingとタンパクの多様化}
\subsection{非コード領域におけるRNA editingとその制御}
\subsection{RNA editingとスプライシングの関係性}
\subsection{RNA editingによる遺伝子発現制御}
\subsection{ADARの発現と細胞内局在}

\section{RNA-seqデータを用いたeditingサイトの検出}
\subsection{検出手法}
\subsection{統計的なフィルタリング手法}
