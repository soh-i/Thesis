\chapter{序論}
\section{はじめに}

\section{RNA editing}
\subsection{RAN editingの種類}
\subsection{ADARのドメイン構造と生化学的機序}
\subsection{細胞内局在と遺伝子発現}

\section{真核生物におけるRNA editing}
\subsection{editingの起こる箇所}
\subsection{位置特異的なediting}
\subsection{乱雑に起こるediting}
\subsection{遺伝子発現制御とeditingの関連性}

\section{RNA-seqデータを用いたeditingサイトの検出}
\subsection{検出手法}
\subsection{統計的なフィルタリング手法}


\chapter{RNA editingサイトの検出手法の開発と実装}

\section{研究背景}

\section{既存の検出手法の定量的評価}
\subsection{評価方法}
\subsection{正解セットの構築}





\section{要求分析}

\section{実装}
\subsection{システムの設計}
\subsection{入力と出力}
\subsection{並列化}
\subsection{解析例}

\section{結果}
\subsection{検出精度のベンチマーク}
\subsection{高速化}

\section{議論}
\subsection{高精度な検出に寄与するパラメータの探索}

\chapter{クマムシにおけるRNA editingサイトの解析}
\section{背景}
\section{解析データと手法}
\subsection{RNA-seqデータのマッピング}
\subsection{検出手法}
\section{結果}
\section{議論}

