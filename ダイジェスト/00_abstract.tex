\begin{jabstract}
RNA編集とは、DNAがRNAへ転写される段階で起こる転写後修飾の一種である。修飾を受けたRNAはゲノムと異なる遺伝情報を持ち、タンパク質機能の多様化や、他の非翻訳RNAとの相互作用を介した遺伝子の発現制御へも関与する。ここ数年、超並列シーケンサーと呼ばれる高出力な塩基配列決定技術が普及し、ヒトやマウスなど高等真核生物に発現するRNAは、高頻度でRNA編集を受けていることが明らかとなってきた。しかしながら、超並列シーケンサーデータを用いたRNA編集サイトの検出には確立された手法が存在していない。
\par
本研究は、上記の問題に対して、これまでに考案されてきたRNA編集サイトの検出手法の精度を比較し、その知見をもとに高精度かつ高速なRNA編集サイトの検出ソフトウェアの開発を行った。
結果、再現率と適合率と呼ばれる指標を導入することにより、既存のRNA編集サイトの検出手法の比較を可能にし、高精度な検出手法と擬陽性を減少させるフィルタリング手法および実験デザインに関する議論を展開する。
本論文は、超並列シーケンサーデータを用いたRNA編集サイトの検出手法の開発に貢献するものである。
\end{jabstract}

\begin{flushright}
	\par
	2014年1月20日
\end{flushright}